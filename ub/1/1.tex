\documentclass[a4paper,twoside,12pt,fleqn]{article}
\usepackage [reqno] {amsmath}
\usepackage{amsfonts,amstext}
\usepackage{amsmath}
\usepackage{amsthm}
\usepackage{german}
\usepackage{graphicx}
\usepackage{fullpage}

\newcommand{\ZETTELNUMMER}{1}
\newcommand{\ABGABEDATUM}{am 27. April 2018 bis 10 Uhr in die
Tutorenf\"acher}

\newcounter{AUFGNR}
\setcounter{AUFGNR}{1}
\newcommand{\AUFGABE}[2]{\vspace{0.3cm}\item[Aufgabe~\arabic{AUFGNR}]\stepcounter{AUFGNR} #1\hfill\emph{#2}}


\newcommand{\floor}[1]{\left\lfloor{#1}\right\rfloor}
\newcommand{\ceil}[1]{\left\lceil{#1}\right\rceil}
\newcommand{\half}[1]{\frac{#1}{2}}

\newcommand{\N}{\mathbb{N}}

% \newtheorem*{ia}{IA}
% \newtheorem*{ib}{IB}
% \newtheorem*{ia}{IA}
\newtheorem*{antwort}{Antwort}

\renewcommand{\labelenumi}{(\alph{enumi})}
\renewcommand{\labelenumii}{(\roman{enumii})}


\begin{document}
\pagestyle{empty}
\hrule\medskip
\rule{0ex}{0ex}\\[-1ex]
\ZETTELNUMMER. Aufgabenblatt zur Vorlesung

\smallskip
\noindent
\large
\textbf{Grundlagen der theoretischen Informatik}\hfill SoSe 2018 \\[0.5ex]
\normalsize
Wolfgang Mulzer, Katharina Klost

\medskip\hrule

\smallskip
\noindent
\textbf{Abgabe} \ABGABEDATUM

\vskip 0.5cm


\begin{description}
\AUFGABE{Vollst\"andige Induktion}{10 Punkte}

In dieser Aufgabe sollen Sie das Prinzip der vollst\"andigen
Induktion wiederholen.

\begin{enumerate}
  \item Sei $n \geq 2$.
    Beweisen Sie durch vollst\"andige Induktion, dass eine Menge mit $n$
    Elementen $n(n-1)/2$ Teilmengen mit genau zwei Elementen hat.

    \begin{enumerate}
      \item Induktionsbehauptung
      \[
        \forall n \in \N, n \geq 2 : \forall A, |A| = n, B = \{\{a_0, a_1\} |
        a_0, a_1 \in A \}: |B| = n (n - 1) / 2
      \]
      \item Induktionsanfang mit $n = 2$\\
      $A = \{x, y\}$ ($x, y$ beliebig)\\
      Die einzige Teilmenge mit 2 Elementen von A ist A selbst: $B = \{A\}$\\
      $|B| = 1$\\
      Nach Behauptung: $|B| = n(n-1)/2 = 2 \cdot 2 / 2 = 1$
      \item Induktionsschritt: $n \rightarrow n + 1$\\
      $|A| = n \Leftrightarrow A = \{a_0, a_1, \dots, a_n\}$\\
      Nach Induktionsbehauptung ist $|B| = n(n-1)/2$\\
      Bekommt die Menge ein weiteres Element $a_{n+1}$, gibt es genau $n$
      neue Teilmengen: $\{a_0, a_{n+1}\} \dots \{a_n, a_{n+1}\}$\\
      Also: $A' = \{a_0, a_1, \dots, a_n, a_{n+1}\}$, $|A'| = n+1$\\
      $|B'| = |B| + n = n(n-1)/2 + n$\\
      Einsetzen Induktionsbehauptung:\\
      $|B'| = (n+1)(n+1-1)/2 = n(n+1)/2$\\
      Zu pr\"ufen: $n(n+1)/2 = n(n-1)/2 + n$
      \begin{align*}
        n(n+1)/2 &= n(n-1)/2 + n\\
                 &= n((n-1)/2 + 1) &| :n \text{ (da $n \geq 2$)}\\
        (n+1)/2  &= (n-1)/2 + 1    &| \cdot 2\\
        n+2      &= n-1 + 2\\
                 &= n + 1
      \end{align*}
    \end{enumerate}
  \item Zeigen Sie durch vollst\"andige Induktion: F\"ur alle
  $n \geq 1$ ist
    \[
      1\cdot 1! + 2 \cdot 2! + \dots + n \cdot n! = (n+1)! -1.
    \]
    \begin{enumerate}
      \item Induktionsbehauptung
      \[
        \forall n \in \N, n \geq 1: 1 \cdot 1! + 2 \cdot 2! + \dots +
        n \cdot n! = (n + 1)! - 1
      \]
      \item Induktionsanfang mit $n = 1$
      \begin{align*}
        1 \cdot 1! &= 1\\
        (1+1)!-1 = 2-1 &= 1
      \end{align*}
      \item Induktionsschritt: $n \rightarrow n + 1$\\
      Zu zeigen: $1\cdot 1! + 2 \cdot 2! + \dots + n \cdot n! \cdot (n+1)(n+1)!
      = ((n+1)+1)!-1$\\
      Einsetzen Induktionsbehauptung:\\
      \begin{align*}
        (n+1)!-1 + (n+1)(n+1)!   &= (n+2)!-1\\
        (n+1)!   + (n+1)(n+1)!   &= (n+2)!\\
        (n+1)! \cdot (1 + (n+1)) &= (n+2)!\\
        (n+1)! \cdot (n+2)       &= (n+2)!\\
        (n+2)!                   &= (n+2)!\\
      \end{align*}
    \end{enumerate}
\end{enumerate}

\AUFGABE{W\"orter}{10 Punkte}

 F\"ur ein \emph{Alphabet} $\Sigma$ mit $k$ Elementen
  ist  $\Sigma^*$ die Menge der W\"orter (Folgen), die
  man aus den Buchstaben von $\Sigma$ bilden kann.
\begin{enumerate}
  \item Wieviele W\"orter in $\Sigma^*$ haben die
    L\"ange $n$ (bestehen aus $n$ Buchstaben)?
    \begin{antwort}
      Es gibt $n$ Buchstaben pro Wort mit jeweils $k$ M\"oglichkeiten, daher gibt
      es $k^n$ m\"ogliche W\"orter in $\Sigma^*$ mit L\"ange $n$.
    \end{antwort}

  \item Wieviele Palindrome in $\Sigma^*$ haben die L\"ange $n$?
(Ein Palindrom ist ein Wort, das von vorne und von hinten gelesen
gleich ist.)
    \begin{antwort}
      $k^{\ceil{\frac{n}{2}}}$
    \end{antwort}
  \item Wieviele W\"orter der L\"ange $n=2,3,4,5,6,7,8$ \"uber dem Alphabet
  $\Sigma = \{\texttt{(},\texttt{)}\}$ sind ``g\"ultige''
  Klammerausdr\"ucke wie zum Beispiel \texttt{(()(()(()))())} und
  \texttt{((()))}? (Un\-g\"ul\-tig sind etwa
  \texttt{(((())} --- mehr offene als geschlossene Klammern, und
  \texttt{())(()} --- die zweite schlie\ss{}ende Klammer hat keine
   \"offnende Partnerklammer.)
   Finden Sie eine (Rekursions-)Formel f\"ur die Anzahl
   der g\"ultigen Klammerausdr\"ucke der L\"ange $n$.
\end{enumerate}

\AUFGABE{Kleene-Stern}{10 Punkte}
\begin{enumerate}
  \item $(A \cup B)^* = A^* \cup B^*$;
    \begin{antwort}
      Falsch: Sei\\
      $A = \{a,c\}$ und\\
      $B = \{b,c\}$ so ist\\
      $acac \in (A \cup B)^*$, jedoch\\
      $acac \notin (A^* \cup B^*)$
    \end{antwort}
  \item $(A \cap B)^* = A^* \cap B^*$;
    \begin{antwort}
      Richtig: Sei\\
      $A = \{a_0, \dots, a_i, c_0, \dots, c_k\}$ und\\
      $B = \{b_0, \dots, b_j, c_0, \dots, c_k\}$, wobei\\
      $a_{0-i} :=$ Elemente aus $A$, die nicht in $B$ sind,\\
      $b_{0-j} :=$ Elemente aus $B$, die nicht in $A$ sind und\\
      $c_{0-k} :=$ Elemente, die in $A$ und $B$ sind.\\
      $A \cap B = \{c_0, \dots, c_k\}$\\
      $(A \cap B)^* = \{\epsilon, c_0, \dots, c_k, c_0c_0, \dots, c_0c_k, \dots\}$\\
      $A^* = \{\epsilon, a_0 \dots a_i, c_0 \dots c_k, a_0a_0 \dots
        a_0c_k \dots c_kc_k \dots\}$\\
      $B^* = \{\epsilon, b_0 \dots b_j, c_0 \dots c_k, b_0b_0 \dots
        b_0c_k \dots c_kc_k \dots\}$\\
      $A^* \cap B^* = \{\epsilon, c_0, \dots, c_k, c_0c_0, \dots, c_0c_k, \dots\}$\\
      $\Rightarrow (A \cap B)^* = A^* \cap B^*$
    \end{antwort}
  \item $(A  \circ B)^* = A^* \circ B^*$;
    \begin{antwort}
      Falsch: Sei\\
      $A = \{a\}$ und\\
      $B = \{b\}$ so ist\\
      $abab \in (A \circ B)^*$, jedoch\\
      $abab \notin (A^* \circ B^*)$
    \end{antwort}
  \item $(A^*)^* = A^*$;
    \begin{antwort}
      Richtig:\\
      \begin{align*}
        A^* &= \bigcup^n_{k=0} A^k = A^0 \cup \dots \cup A^n\\
        (A^*)^* &= \{\epsilon\} \cup (A^0\cup\dots\cup A^n) \cup
          (A^0\cup\dots\cup A^n)^2 \cup \dots \cup (A^0\cup\dots\cup A^n)^m\\
          &= \{\epsilon\} \cup A^0 \cup \dots \cup A^n
      \end{align*}
    \end{antwort}
  \item $A \subseteq B \Rightarrow A^* \subseteq B^*$.
    \begin{antwort}
      Richtig. Sei\\
      $A = \{a_0, \dots, a_i\}$\\
      $B = \{a_0, \dots, a_i, b_0, \dots, b_j\}$\\
      $a_{0-i}$ := Elemente aus $A$ und $B$\\
      $b_{0-j}$ := Elemente aus $B$, die nicht in $A$ sind\\
      $\Rightarrow A \subseteq B$\\
      $A^* = \{\epsilon, a_0 \dots a_i, a_0a_0 \dots a_ia_i \dots\}$\\
      $B^* = \{\epsilon, a_0 \dots a_i, b_0 \dots b_j,
        a_0a_0 \dots a_ia_i \dots a_ib_0 \dots a_ib_j
        \dots b_0a_0 \dots\}$\\
      $\Rightarrow A^* \subseteq B^*$
    \end{antwort}
\end{enumerate}
\end{description}
\end{document}