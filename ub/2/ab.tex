\documentclass[a4paper,twoside,12pt,fleqn]{article}
\usepackage [reqno] {amsmath}
\usepackage{amsfonts,amstext}
\usepackage{amsmath}
\usepackage{amsthm}
\usepackage{german}
\usepackage{graphicx}
\usepackage{fullpage}

\newcommand{\ZETTELNUMMER}{2}
\newcommand{\ABGABEDATUM}{am 4. Mai 2018 bis 10 Uhr in die
Tutorenf\"acher}

\newcounter{AUFGNR}
\setcounter{AUFGNR}{1}
\newcommand{\AUFGABE}[2]{\vspace{0.3cm}\item[Aufgabe~\arabic{AUFGNR}]\stepcounter{AUFGNR} #1\hfill\emph{#2}}


\newcommand{\floor}[1]{\left\lfloor{#1}\right\rfloor}
\newcommand{\ceil}[1]{\left\lceil{#1}\right\rceil}
\newcommand{\half}[1]{\frac{#1}{2}}

\newcommand{\N}{\mathbb{N}}
\newtheorem*{antwort}{Antwort}


\renewcommand{\labelenumi}{(\alph{enumi})}
\renewcommand{\labelenumii}{(\roman{enumii})}


\begin{document}
\pagestyle{empty}

\noindent
\large
\textbf{Grundlagen der theoretischen Informatik}\hfill SoSe 2018 \\[0.5ex]
\normalsize
Anton Oehler, Jona Rex

\medskip\hrule

\smallskip
\noindent
\textbf{Abgabe} \ABGABEDATUM



\begin{description}
	\AUFGABE{Regul\"are Ausdr\"ucke I}{10 Punkte}

	\begin{enumerate}
		\item
		      Geben Sie regul\"are Ausdr\"ucke f\"ur die folgenden Sprachen an.
		      Erkl\"aren Sie Ihren Ausdruck in einem Satz, und geben Sie das zu
		      Grunde liegende Alphabet an. Erkl\"aren Sie ggf.\@ auch alle
		      Annahmen, die Sie gemacht haben.
		      \begin{enumerate}
			      \item Summen von positiven dezimalen Festkommazahlen.
			            In der Sprache sollen also z.B.\@ die Zeichenketten
                  $3.14$ oder auch $3+4.2+7+1$ vorkommen.
                  \begin{align*}
                    \Sigma &= \{0, \dots, 9, `.`, +\}\\
                    \text{Sei } d &= \{0,1,2,3,4,5,6,7,8,9\}\\
                    \text{und } f &= d \cup dd\\
                    R &= d^+\{`.` \circ f, \epsilon\} \> (+ \circ d^+\{`.` \circ f, \epsilon\})^*
                  \end{align*}
                  Eine Summe von positiven dez. Festkommazahlen besteht aus einer ersten Zahl
                  und optional einer beliebig langen Folge von `+` und weiteren Zahlen.
                  Jede Zahl setzt sich aus mindestens einer Ziffer zusammen, optional gefolgt
                  von einem Punkt (hier der Sichtbarkeit halber $`.`$) und einer oder zwei
                  weiteren Ziffern.\\
                  \textbf{Annahme:} Festkommazahlen beschr"anken sich auf 2 Nachkommastellen, sonst ist
                  $f = d^+$ und auch der Teil vor dem Komma darf mit $0$ beginnen.
			      \item KFZ-Nummernschilder, bestehend aus ein bis drei Gro\ss{}buchstaben
			            gefolgt von einem Bindestrich, dann ein bis
			            zwei Gro\ss{}buchstaben und zum Schluss ein bis vier Ziffern.
			            Die L\"ange der Nummernschilder insgesamt betr\"agt nicht mehr als
                  acht Zeichen.
                  \begin{align*}
                    \Sigma &= \{A, \dots, Z, 0, \dots, 9, -\}\\
                    \text{Sei } a &= \{A, \dots, Z\}\\
                    \text{und } d &= \{0, \dots, 9\}\\
                    R = \> &(a-(a \cup aa)(d \cup dd \cup ddd \cup dddd)) \cup\\
                           &(aa-a(a \cup d \cup \epsilon)(d \cup dd \cup ddd)) \cup\\
                           &(aaa-a(a \cup d \cup \epsilon)(d \cup dd)
                  \end{align*}
                  Wir betrachten 3 F"alle: (hier: Buchstabe = Gro"sbuchstabe)
                  \begin{itemize}
                    \item 1 Buchstabe am Anfang, gefolgt von einem Bindestrich und
                    1-2 Buchstaben und 1-4 Zahlen
                    \item 2 Buchstaben am Anfang, gefolgt von einem Bindestrich und einem
                    Buchstaben. Dann folgt [entweder ein Buchstabe, eine Zahl oder nichts]
                    und danach 1-3 Zahlen
                    \item 3 Buchstaben am Anfang, ein Bindestrich, gefolgt von einem Buchstaben,
                    gefolgt von [entweder einem Buchstaben, einer Zahl oder nichts] und
                    anschlie"send 1-2 Zahlen
                  \end{itemize}
                  \textbf{Annahme:} Der Bindestrich z"ahlt auch als Zeichen, nicht nur Buchstaben
                  und Zahlen.
			      \item Die Menge aller Bin\"arzahlen, die durch vier teilbar sind.
                  Dabei sind keine \"uberfl\"ussigen f\"uhrenden Nullen erlaubt.
                  \begin{align*}
                    R &= 0 \cup (1(0 \cup 1)^* \cup \epsilon)100
                  \end{align*}
                  Entweder $0$, $100$ ($\epsilon \circ 100$) oder eine Zahl die mit $1$ beginnt,
                  gefolgt von beliebigen Kombinationen von $0$ und $1$, gefolgt von $100$.
		      \end{enumerate}
		\item
		      Beschreiben Sie in deutscher Sprache die durch die folgenden regul\"aren
		      Ausdr\"ucke charakterisierten Mengen.
		      Ihre Beschreibungen sollen die Form haben:
		      \begin{center}
			      \emph{Die Menge aller W\"orter \"uber dem
				      Alphabet $\{0,1\}$ \dots .``}  %''
		      \end{center}
		      Dabei soll \dots durch maximal acht W\"orter ersetzt werden.

		      \begin{enumerate}
            \item $0^*(0^* 1 0^* 1 0^*)^*$
            Beliebige Folge mit gerader Anzahl an Einsen
			      \item $(00 \cup 11 \cup (01 \cup 10) (00 \cup 11)^* (01 \cup 10))^*$.
		      \end{enumerate}
	\end{enumerate}

	\AUFGABE{Regul\"are Ausdr\"ucke II}{10 Punkte}

	Zwei regul\"are Ausdr\"ucke $\alpha, \beta$ hei\ss{}en \emph{\"aquivalent},
	geschrieben $\alpha \sim \beta$, genau dann, wenn sie die gleiche
	Sprache repr\"asentieren, also $L(\alpha)=L(\beta)$ ist.
	\renewcommand{\a}{\alpha}
	\renewcommand{\b}{\beta}
	\newcommand{\e}{\epsilon}
	Zeigen Sie:
	\begin{enumerate}
		\item $(\alpha \cup \beta)\gamma \sim \alpha\gamma \cup \beta\gamma$\\
		Zu zeigen:
		\begin{align*}
			L((\alpha\cup\beta)\gamma) &= L(\alpha\gamma\cup\beta\gamma)\\
			L(\alpha\cup\beta)\circ L(\gamma) &= L(\alpha\gamma)\cup L(\beta\gamma)\\
			\{\alpha, \beta\}\circ\{\gamma\} &= \{\alpha\circ\gamma\}\cup\{\beta\circ\gamma\}\\
			\{\alpha\circ\gamma, \beta\circ\gamma\} &= \{\alpha\circ\gamma, \beta\circ\gamma\} &\Box
		\end{align*}
		\item $(\alpha^*)^* \sim \alpha^*$\\
		Zu zeigen:
		\begin{align*}
			L((\alpha^*)^*) &= L(\alpha^*)\\
			L(\a^*) &= \{\epsilon,\a,\a\a,\a\a\a,\dots\}
		\end{align*}
		Konkateniert man nun f"ur $L((\a^*)^*)$ alle W"orter aus
		$\{\epsilon,\a,\a\a,\a\a\a,\dots\}$ so kann dadurch nur wieder
		$\{\epsilon,\a,\a\a,\a\a\a,\dots\}$ entstehen:
		\begin{align*}
			\e\circ\e &= \e\\
			\e\circ\a &= \a\\
			\e\circ\a\a &= \a\a\\
			\dots\\
			\a\circ\e &= \a\\
			\a\circ\a &= \a\a\\
			\dots
		\end{align*}
		Also ist $L(\a^*)^2 = L(\a^*)$ und nach gleichem Schema auch\\
		\[ \forall k\in\N, k>0: L(\a^*)^k = L(\a^*) \]
		und f"ur $k = 0: L(\a^*)^k = L(\a^*)^0 = \{\e\}$. Daraus folgt:\\
		\begin{align*}
			L(\a^*)^* &= \bigcup_{k=0}^\infty L(\a^*)^k\\
			&= L(\a^*)^0 \>\cup \bigcup_{k=1}^\infty L(\a^*)^k\\
			&= \{\e\} \>\cup \bigcup_{k=1}^\infty L(\a^*)^k\\
			&= \{\e\} \>\cup \bigcup_{k=1}^\infty L(\a^*)\\
			&= \{\e\} \>\cup L(\a^*)\\
			&= L(\a^*) &\Box
		\end{align*}
		\item $(\alpha \cup \beta)^* \sim (\alpha^*\beta^*)^*$;
		\item $\alpha\emptyset \sim \emptyset$\\
			Zu zeigen:
			\begin{align*}
				L(\a\emptyset) &= L(\emptyset)\\
				\{w \circ v | w \in \{ \a \} \wedge v \in \emptyset\} &= \emptyset\\
				\emptyset &= \emptyset &\Box
			\end{align*}
		\item $\alpha\emptyset^* \sim \alpha$\\
			Zu zeigen:
			\begin{align*}
				L(\a\emptyset^*) &= L(\a)\\
				L(\a)\circ L(\emptyset^*) &= L(\a)
			\end{align*}
			Per Definition ist $\emptyset\circ\emptyset=\emptyset$ und
			$\emptyset\cup\emptyset=\emptyset$. Somit ist
			\begin{align*}
				\emptyset^* &= \emptyset^0 \cup \emptyset^1 \cup \emptyset^2 \dots\\
				&= \emptyset^0 \cup \emptyset \cup \emptyset \dots\\
				&= \{\e\}
			\end{align*}
			Also ist
			\begin{align*}
				L(\a) \circ L(\emptyset\} = L(\a) \circ \{\e\} = L(\a) &&\Box
			\end{align*}
	\end{enumerate}
	\newpage

	\AUFGABE{Regul\"are Sprachen}{10 Punkte}

	Eine Sprache hei\ss{}t \emph{regul\"ar}, falls sie sich durch einen
	regul\"aren Ausdruck darstellen l\"asst.

	\begin{enumerate}
	\item Zeigen Sie: Jede endliche Sprache ist regul\"ar.\\
	\textbf{Antwort: }\\
	Eine endliche Sprache besteht nur aus endlich vielen W"ortern.
	Eine (primitive) Darstellung dieser Sprache ist, alle m"oglichen
	W"orter miteinander zu vereinigen.
	\item
	Sei $\Sigma$ ein Alphabet und $w = \sigma_1\sigma_2\dots\sigma_n$
	ein Wort \"uber $\Sigma$. Die \emph{Umkehrung} von $w$, $w^R$, ist
	definiert als $w^R = \sigma_{n}\sigma_{n-1}\dots\sigma_1$.
	So ist zum Beispiel $\text{haus}^{R}=\text{suah}$,
	$\text{blatt}^{R}=\text{ttalb}$,
	$\text{a}^{R}=\text{a}$ und
	$\varepsilon^{R}=\varepsilon$.
	Die Umkehrung einer Sprache $L \subseteq \Sigma^*$
	ist definiert als
	$L^R = \{w^R \mid w \in L\}$.

	Zeigen Sie: Ist $L$ eine regul\"are Sprache, so ist auch
	$L^R$ regul\"ar. Geben Sie dazu einen detaillierten Beweis,
	der strukturelle Induktion \"uber den regul\"aren Ausdruck
	f\"ur $L$ verwendet.\\
	\textbf{Antwort: }\\
	Sei $w$ ist Wort "uber $\Sigma$ mit $w = o_1o_2\dots o_n$ und\\
	$w^R$ seine Umkehrung mit $o_no_{n-1}\dots o_1$.\\
	Sei die Umkehrung einer Sprache $L \subseteq \Sigma^*$ definiert als
	$L^R = \{w^R | w \in L\}$.\\
	% Im folgenden bedeuted $L_n \Leftrightarrow |L| = n$
	\begin{enumerate}
		% \item[Voraussetzung:] $L_n$ ist regul"ar $\Rightarrow L_n^R$ ist regul"ar
		\item[Voraussetzung:] $L$ ist regul"ar $\Rightarrow L^R$ ist regul"ar
		\item[Anfang:]
			mit $L = \emptyset$:\\
			Aus der Definition von $L^R = \{w^R | w \in L\}$ folgt:
			\[ L = \emptyset \> \Leftrightarrow L^R = \emptyset \]
			mit $L = \{\epsilon\}$:\\
			Per Definition ist $\epsilon^R = \epsilon$. Also gilt:
			\[ L = \{\epsilon\} \> \Leftrightarrow L^R = \{\epsilon\} \]
		\item[Schritt:] Laut \emph{IV.} gilt:\\
			Wenn $L$ regul"ar, dann ist auch $L^R$ regul"ar.\\
			Sei nun $w = o_1o_2\dots o_n$ ein beliebiges
			Wort der L"ange $n$ in $L$ und\\
			$w^R = o_no_{n-1}\dots o_1$
			seine Umkehrung in $L^R$.

			Zu zeigen: f"ur $a \in \Sigma$ gilt:
			\begin{itemize}
				\item $(w \circ a) \in L'$,($L'$ reg.) $\Rightarrow
					(w\circ a)^R \in L'^R$,($L'^R$ reg.)
					\begin{align*}
						&(w\circ a) = o_1o_2\dots o_na\\
						\Rightarrow &(w\circ a)^R = ao_no_{n-1}\dots o_2o_1\\
						\Rightarrow &L'^R \text{ ist regul"ar}
					\end{align*}
				\item $(w\cup a) \in L'$,($L'$ reg.) $\Rightarrow
					(w\cup a)^R \in L'^R$,($L'^R$ reg.)
					\begin{align*}
						&(w\cup a) = \{w, a\}\\
						\Rightarrow &(w\cup a)^R = \{w^R, a^R\}\\
						\Rightarrow &(w\cup a)^R = \{w^R, a\}
					\end{align*}
					Dass $w^R \in L'^R$ l"asst sich wie zuvor zeigen, und da
					$a^R = a$ (Definition), ist $(w\cup a)^R \in L'^R$ und
					$L'^R$ regul"ar.
			\end{itemize}
	\end{enumerate}
	\end{enumerate}

\end{description}
\end{document}
